\documentclass{ltjsarticle} %lualatex cs_jikken.texで作成 
\usepackage{mdframed}
\usepackage{graphicx}
\usepackage{float}
\usepackage{array}
\usepackage{tikz}

\usepackage{circuitikz}

\begin{document}

\begin{circuitikz} \draw
  % Raspberry Pi GPIO Pins
  (0,0) node[anchor=east]{3.3V} -- (2,0)
  (0,-0.5) node[anchor=east]{GPIO 2 (SDA)} -- (2,-0.5)
  (0,-1) node[anchor=east]{GPIO 3 (SCL)} -- (2,-1)
  (0,-1.5) node[anchor=east]{GND} -- (2,-1.5)
  (0,-2) node[anchor=east]{GPIO 17} -- (2,-2)
  (0,-2.5) node[anchor=east]{GPIO 18} -- (2,-2.5)
  (0,-3) node[anchor=east]{GPIO 27} -- (2,-3)
  (0,-3.5) node[anchor=east]{GPIO 22} -- (2,-3.5);
  
  % Example: Connecting a LED and resistor to GPIO
  (2,-2) to[R, l=$1k\Omega$] (4,-2)  % Resistor
         to[led, l=LED] (6,-2)       % LED
         to[short] (6,-1.5)          % Connect to GND
         to[short] (2,-1.5);
\end{circuitikz}


\begin{circuitikz} \draw
  % Raspberry Pi GPIO Pins (3V3, GPIO 17, GND)
  (0,0) node[anchor=east]{3.3V} -- (2,0)
  (0,-1) node[anchor=east]{GPIO 17} -- (2,-1)
  (0,-2) node[anchor=east]{GND} -- (2,-2)

  % LED with resistor connected to GPIO 17
  (2,-1) to[R, l=$330\Omega$] (4,-1)  % Resistor
         to[led, l=LED] (6,-1)        % LED
         to[short] (6,-2)             % Connect to GND
         to[short] (2,-2);            % Closing the circuit
\end{circuitikz}

\end{document}